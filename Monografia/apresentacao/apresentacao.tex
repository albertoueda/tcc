\documentclass{beamer}
\usepackage[brazil]{babel}
\usepackage[utf8x]{inputenc} 
\PrerenderUnicode{ç}
\setbeamercovered{transparent=5}
\usetheme{Copenhagen}
\title{Physimulation: Biblioteca Gráfica para Simulações de Física}
\author{Alberto Hideki Ueda e Rafael Issao Miyagawa\\ \url{rafaelim@ime.usp.br} \url{alberto@ime.usp.br}}

\institute{Instituto de Matemática e Estatística\\Universidade de São Paulo}

\begin{document}

\frame{\titlepage}
\frame{\tableofcontents}
\section{Introdução}
\begin{frame}
  \frametitle{Motivação}
  \begin{itemize}
    \item Por que temos disciplinas como física e estatística?
    \item Como relacionar com a computação
    \item Exemplos
  \end{itemize} 
\end{frame}
\begin{frame}
  \frametitle{Objetivo}
  \begin{itemize}
    \item Criar um simulador físico para aulas de física
    \item Problemas e soluções para simulações de ambientes físicos
  \end{itemize}
\end{frame}
\section{Ferramentas e Conceitos}
\begin{frame}
  \frametitle{Linguagem e Plataformas}
  \begin{block}{Ruby}
    Linguagem dinâmica e orientada a objetos
  \end{block}
  \begin{block}{Chipmunk}
    Biblioteca física voltada para jogos
  \end{block}
  \begin{block}{Gosu + Chingu}
    Toolkit para criar interfaces de jogos
  \end{block}
  \begin{block}{Glade}
    User Interface Designer
  \end{block}
\end{frame}
\begin{frame}
  \frametitle{Conceitos gerais}
  \begin{block}{Tempo de simulação}
    \begin{itemize}
      \item Fixo
      \item Variável
      \item Semi-Fixo
    \end{itemize}
  \end{block}
  \begin{block}{Broad Phase}
    \begin{itemize}
      \item Sweep and Prune
      \item AABB tree
      \item Spatial Hashing
    \end{itemize}
  \end{block}
  \begin{block}{Detecção de colisão (Narrow Phase)}
    \begin{itemize}
      \item Separating Axis Theorem
    \end{itemize}
  \end{block}
\end{frame}
\section{Resultados e Demonstrações} 

\begin{frame}
  \frametitle{Physimulation}
    \begin{itemize}
      \item Demonstrações físicas em formato de jogo
    \end{itemize} 
\end{frame}

\section{Conclusão}

\begin{frame}
  \frametitle{Conclusão}
  \begin{itemize}
    \item
    \item
    \item
    \item
  \end{itemize}
\end{frame}

\begin{frame}
  \frametitle{Próximos passos}
  \begin{itemize}
    \item
  \end{itemize}
\end{frame}

\begin{frame}
  \frametitle{Agradecimentos}
  \begin{itemize}
    \item Prof. Dr. Jose Coelho Pina
    \item João Pedro Kerr Catunda
  \end{itemize}
\end{frame}

\end{document}
