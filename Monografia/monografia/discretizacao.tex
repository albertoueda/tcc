A simulação no Chipmunk ocorre em intervalos de tempo, ou seja, precisamos indicar o tempo em que os objetos vão interagir como vimos na seção \ref{plataforma}. 
Esse intervalo que passamos para o Chipmunk é o tempo de simulação. É preciso definir com atenção esse valor para que a simulação não fique muito rápida
ou muito lenta. Embora pareça trivial, há diferentes implementações que indicam como definir esse tempo de simulação. Nas 
próximas seções apresentaremos as implementações para tempo de simulação fixo e variável.

\subsection{Fixo}

A maneira simples é fixar o tempo de simulação, como por exemplo 1/60 de um segundo:

\begin{lstlisting}[language=Ruby, caption=Implementação de tempo de simulação fixo]
def dt = 1/60

while (running) do
  $space.step(dt)
end
\end{lstlisting}

\ \\
\hspace*{14pt} Na maioria das simulações o código acima é o ideal. Se a simulação coincidir com a taxa de atualização de quadro (\textit{frame rate}) e ainda se a chamada {\tt step(dt)} não demorar mais do que 1/60 de um segundo, então a solução é bastante satisfatória. Mas no mundo real não sabemos
o valor da taxa de atualização do quadro. Por exemplo, a execução de uma simulação em uma máquina mais lenta, que não pode atualizar os quadros em 60 fps (\textit{frames per second}),
pode deixar a simulação mais lenta ou mais rápida. \\

Uma solução para evitar esse problema é o tempo de simulação variável.

\subsection{Variável}

A idéia do tempo de simulação variável é medir quanto tempo leva o quadro anterior e, em seguida, alimentar a simulação com esse tempo: 

\newpage
\begin{lstlisting}[language=Ruby, caption=Implementação de tempo de simulação variável]
def current_time = Time.now

while (running) do
  new_time = Time.now
  dt = get_simulation_time(new_time - current_time)

  $space.step(dt)
end

\end{lstlisting}

\ \\
\hspace*{14pt} A solução é simples e aparentemente razoável. Em um computador com taxa de atualização de quadro de 30 fps é considerado o tempo de {1/30} de um segundo para a simulação.
Mas existe um problema com essa abordagem: a simulação não garante resultados corretos para todos os valores de tempo de simulação. 
Dependendo do valor do tempo e da velocidade dos objetos, poderemos observar objetos atravessando paredes ou outros objetos, diminuindo o realismo e o valor da animação.

\subsection{Tempo de Simulação no Physimulation}

Muitos estudos mostram que tempo de simulação fixo é indicado para simulações físicas e tempo de simulação variável para jogos, ou seja, quando a velocidade da animação for mais 
importante que a própria simulação. Porém, no Physimulation, tanto a simulação quanto a animação são importantes. \\
 
No nosso caso, o tempo de simulação fixo foi suficiente pois não afetou consideravelmente a animação. Consequentemente, o Physimulation pode ficar mais lento em computadores com menor poder de processamento.
