
O foco deste projeto é atrair a atenção dos alunos do IME em relação às disciplinas de física ministradas no curso de bacharelado em Ciência da Computação.

\subsection{Física no BCC}
Embora seja comum encontrar alunos da computação que tenham interesse em tópicos relacionados a física clássica e moderna - Leis de Newton, Leis de Kepler, Teoria da Relatividade de Einstein, entre outros - há considerável resistência do corpo discente em relação a grade curricular do BCC incluir as disciplinas de física. (TODO citar questionário BCC) Isto não se aplica apenas às disciplinas de física, mas também a álgebra, cálculo, probabilidade e estatística (TODO referência). \\

Visando compreender melhor a situação e apresentar uma proposta de reformulação da grade ou das disciplinas, foi criado o Grupo de Apoio ao BCC (TODO nomes), uma iniciativa da Comissão de Coordenação do BCC (CoC). Após as discussões a e aplicação de questionários nos Encontros BCC (TODO citar), um dos consensos de ambas as partes - professores e alunos - foi o de que seria interessante uma maior contextualização destas disciplinas para os alunos em termos de computação. No apêndice da seção \ref{apendice} há uma cópia de um dos textos produzidos pelo CoC, com referências para o documento original. \\

Tal contextualização pode ser feita de várias formas: por parte dos professores das disciplinas - mostrando aplicações dos tópicos estudadas para computação, trazendo professores ou especialistas em sala de aula que possam transmitir experiências relacionadas ao conteúdo aprendido - ou por parte dos próprios alunos - reunindo-se para palestras extra-classe, desenvolvendo bibliotecas que motivem os demais alunos a utilizar, pesquisar e até mesmo programar baseando-se nos conceitos adquiridos. \\

Nesse sentido, o Physimulation é uma ferramenta para simulação e animação de fenômenos físicos e tem como um dos principais objetivos integrar o conhecimento adquirido em FAP-126 (Física I) com um ambiente de programação (TODO programa?). \\

Com este trabalho visamos contextualizar os assuntos abordados na disciplina com demonstrações de ambientes físicos, integrações com exercícios-programas (EP's) e a apresentação de desafios atuais de simulação no campo da física. 

\subsection{Problemas ao simular interações físicas}
Existem muitos problemas ao tentar simular um ambiente físico com a computação. Temos o problema do tempo de simulação, 
que é o tempo que damos para os objetos físicos se interagirem e tomar o rumo necessário para refletir a realidade. \\

Existem alguns conceitos como \textit{broad phase}, que é a fase em que os objetos são filtrados para realizarmos depois a \textit{narrow phase} que é onde verificamos se aconteceu alguma colisão entre os objetos.

\subsection{Simulador}
Com o Physimulation, o aluno da disciplina de física poderá simular comportamentos físicos estudados em sala de aula, por meio de criação de cenários e objetos com diferentes propriedades configuráveis. Exemplos de propriedades são: massa, tamanho, posição, velocidade inicial, coeficiente de atrito e de elasticidade. \\

Desde rampas, pequenas esferas e plataformas fixas a planetas orbitando em torno de um objeto de grande massa, tanto os alunos quanto o professor poderão utilizar o simulador para combinar tais objetos e observar resultados calculados em exercícios. Além disso, estes cenários podem ser guardados no sistema e carregados posteriormente para nova observação.

\subsection{Estrutura da Monografia}
Esta monografia está estruturada da seguinte forma: Na seção \ref{plataforma} descrevemos os conceitos e tecnologias estudadas para a realização do trabalho. São apresentadas as bibliotecas utilizadas, o papel de cada uma em nosso projeto e como elas interagem entre si. Nas seções \ref{discretizacao} e \ref{colisoes} explicamos dois tópicos comuns e importantes ao realizar simulações físicas: são eles, respectivamente, a discretização da simulação - definição de tempo de simulação, problema do tamanho do passo, granularidade - e o tratamento de colisões - algoritmos para filtragem de elementos, detecção de colisões e quais algoritmos são utilizados no Physimulation.  \\

Já na seção \ref{atividades} mostramos como foi o passo-a-passo de nosso trabalho, a metodologia utilizada e algumas dificuldades encontradas no caminho. A partir da seção \ref{animacoes}, apresentamos um dos resultados obtidos com o trabalho: primeiramente, os ambientes físicos de demonstração e as razões por ter começado o trabalho por eles. Logo depois, na seção \ref{physimulation} descrevemos o Physimulation e as etapas para a criação de cenários físicos. Como aplicação do projeto, apresentamos duas integrações com exercícios-programas de disciplinas de introdução a computação na seção \ref{ep}. \\

Concluímos o trabalho na seção \ref{comentarios}, além de citar possíveis trabalhos futuros e mostrar desafios atuais da computação em simulações físicas que ainda necessitam de estudo e aprimoramento. Ao final, há um roteiro para instalação do ambiente necessário para executar o Physimulation, na seção \ref{instalacao}. 


