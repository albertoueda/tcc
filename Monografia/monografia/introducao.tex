
\subsection{Física no BCC}

A disciplina de Física (FAP-0126), oferecida no curso de BCC, é puramente teórica e não mostra nenhuma relação com a Ciência da Computação. Isso torna a disciplina menos interessante e frequentemente faz os alunos pensarem: "Para que serve esta disciplina?".
Para motivar os alunos e ilustrar melhor a relação entre as disciplinas básicas (Física, Estatística, Álgebra e Cálculo) com a Ciência da Computação, pretendemos criar uma biblioteca gráfica de simulação. Esta biblioteca será capaz de realizar uma leitura de dados de uma simulação de um EP e mostrar graficamente o resultado da simulação, por exemplo.
Esta biblioteca também proporcionará um ambiente de simulação específico e pronto para ser mostrado em salas de aula.

O foco deste projeto é atrair a atenção dos alunos do IME em relação a disciplina de física ministrada no curso de bacharelado em ciência da computação.
Com o simulador podemos integrar melhor os alunos aos assuntos abordados na física com demonstrações de ambientes físicos, 
integrando exercícios programas (EP) e também para ser utilizado em sala de aula.

\subsection{Problemas ao simular interações físicas}

Existem muitos problemas ao tentar simular um ambiente físico com a computação. Temos o problema do tempo de simulação, 
que é o tempo que damos para os objetos físicos se interagirem e tomar o rumo necessário para refletir a realidade.
Existem alguns conceitos como broad phase, que é a fase em que os objetos são filtrados para realizarmos depois a narrow phase que é onde verificamos se aconteceu 
alguma colisão entre os objetos.

\subsection{Simulador}

A idéia do simulador é poder mostrar os problemas que encontramos ao tentar simular um ambiente físico com montagem de demonstrações. 

\subsection{Estrutura da Monografia}

Na seção 1.2, apresentamos...h
