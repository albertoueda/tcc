
O foco deste projeto é atrair a atenção dos alunos do IME em relação às disciplinas de física ministradas no curso de bacharelado em Ciência da Computação.

\subsection{Física no BCC}
Embora seja comum encontrar alunos da computação que tenham interesse em tópicos relacionados a física clássica e moderna - Leis de Newton, Leis de Kepler, Teoria da Relatividade de Einstein, entre outros - há considerável resistência do corpo discente em relação a grade curricular do BCC incluir as disciplinas de física. (TODO citar questionário BCC, confirmar) Isto não se aplica apenas às disciplinas de física, mas também a álgebra, cálculo, probabilidade e estatística (TODO referência).

Visando compreender melhor a situação e apresentar uma proposta de reformulação da grade ou das disciplinas, foi criado o Grupo de Apoio ao BCC (TODO nomes), uma iniciativa da Comissão de Coordenação do BCC (CoC). Após a aplicação de questionários e conversas nos Encontros BCC (TODO citar), um dos consensos de ambas as partes - professores e alunos - foi o de que seria interessante uma maior contextualização destas disciplinas para os alunos em termos de computação. No apêndice da seção \ref{apendice} há uma cópia de um textos produzidos pelo CoC, com referências para o documento original. 

Nesse sentido... (TODO falar de como a fisica pode ficar mais interessante e nossa ideia)

Com o simulador podemos integrar melhor os alunos aos assuntos abordados na física com demonstrações de ambientes físicos, 
integrando exercícios programas (EP) e também para ser utilizado em sala de aula.

\subsection{Problemas ao simular interações físicas}

Existem muitos problemas ao tentar simular um ambiente físico com a computação. Temos o problema do tempo de simulação, 
que é o tempo que damos para os objetos físicos se interagirem e tomar o rumo necessário para refletir a realidade.

Existem alguns conceitos como \textit{broad phase}, que é a fase em que os objetos são filtrados para realizarmos depois a \textit{narrow phase} que é onde verificamos se aconteceu alguma colisão entre os objetos.

\subsection{Simulador}

A idéia do simulador é... 

\subsection{Estrutura da Monografia}

Na seção 1.2, apresentamos...
