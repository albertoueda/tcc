O texto a seguir foi extraído do site: \url{http://bcc.ime.usp.br}. \\

{\large \textbf{Material de Apoio ao BCC}} \\

A falta de contextualização das disciplinas básicas do BCC tem sido uma queixa recorrente do alunos nas reuniões entre alunos e professores, no Encontro do BCC de 2010 e também no processo de avaliação semestral que é realizado pelo orientador pedagógico orientador pedagógico da Escola Politécnica (POLI), Giuliano Salcas Olguin.

A fim de motivar os alunos e ilustrar a relação entre ciência da computação e as disciplinas básicas de álgebra, cálculo, estatística, probabilidade e física presentes no currículo do BCC a CoC sugeriu que fossem produzidos documentos ilustrando aplicação de cada uma dessas disciplinas em ciência da computação e vice-versa.  Esses documentos têm o objetivo de motivar os alunos do BCC:

\begin{enumerate}
\item ilustrando as relações entre as disciplinas básicas do curso e ciência da computação;
\item mostrando aos alunos quais das disciplinas mais avançadas do BCC que fazem uso dos conteúdos das disciplinas básicas;
\item fornecendo aos professores das disciplinas básicas do BCC exemplos de aplicações de suas especialidades em ciência da computação, que, eventualmente, podem ser mencionados em aulas ou ser temas de trabalhos.
\end{enumerate}

Esses documentos poderão também ser usados pelas disciplinas de Introdução à Ciência da Computação que são oferecidas pelo DCC para várias unidades da USP. Nestas disciplinas, frequentemente, os chamados exercícios programas ilustram aplicações de métodos computacionais na solução de problemas em genômica, física, economia, etc.
Por exemplo, na última edição da disciplina \begin{quote} MAC2166 Introdução à Ciência da Computação para Engenharia \end{quote}
podemos ver um exercício programa em que é simulada a "trajetória de livre de retorno" de uma nave sob a ação gravitacional da Terra e da Lua em \url{http://www.ime.usp.br/~mac2166/ep3/}. Já um exercício programa com aplicação em genômica pode ser visto em \url{http://www.ime.usp.br/~mac2166/ep4/}.

Além de uma maior integração do curso este projeto pretende propor possíveis mudanças na grade curricular do BCC. Para isto pretendemos realizar uma pesquisa com o egresso do BCC e uma pesquisa das grades curriculares dos cursos de computação pelo mundo.
