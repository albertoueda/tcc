Nesta seção descreveremos a relação entre nosso projeto e a experiência adquirida no BCC.

(TODO pensei em fazer separado mas agora que escrevi tenho impressão que não vai mudar muita coisa)

\section{Alberto Ueda}
Entregar este projeto como trabalho de formatura e disponibilizar seu código para os alunos do BCC foram duas das experiências mais gratificantes que já tive. Isto pois acredito que tal conteúdo poderá ser utilizado pelas próximas turmas do BCC como incentivo ao aprendizado da matéria de física. Além disso, tanto alunos do próprio Instituto de Física quanto da Engenharia Politécnica também poderão se interessar pelo conteúdo: o primeiro grupo (FIS) pela animação de fenônemos físicos estudados e o segundo (Poli) tanto pela animação quanto pela simulação de tais fenônemos.

Mas, ao mesmo tempo, por ser um trabalho que levou meses, certas dificuldades foram encontradas pelo caminho. Tivemos que tomar decisões às vezes frustrantes, porém necessárias.

\subsection{Desafios e frustrações encontrados}
Inicialmente, nossa motivação era entregar um sistema que utilize recursos do Wii Remote (TODO ref TODO link da caneta) e que o professor pudesse utilizá-lo em sala de aula para realizar suas simulações e animações. Porém, chegamos a conclusão que esta tecnologia aumentaria consideravelmente o nível de complexidade de nosso trabalho e não tínhamos garantia de que utilizá-la acrescentaria da mesma forma ao resultado final. Assim descartamos esta possibilidade.

Como utilizamos algumas bibliotecas de terceiros em nosso projeto, tivemos que entender obrigatoriamente como eram feitas as principais chamadas de métodos destas bibliotecas, principalmente o Chipmunk e o Gosu. Um detalhe interessante que ocorreu no segundo mês de trabalho foi a necessidade de mudar o código da biblioteca (TOOD citar) e recompilá-la para que uma função simples de mensagem para o usuário funcionasse (TODO conferir método). Uma semana depois, utilizando uma versão mais nova da biblioteca, descobrimos que nossa alteração não era mais necessária, pois já havia sido feita pelos próprios programadores na mudança de versão.

Além disso, utilizamos um binding (TODO envoltório?) da versão original do Chipmunk. Isto trazia duas dificuldades para nós: 1) o código original (em C++) sempre estava com uma versão mais recente e provia (TODO conferir) mais métodos; e 2) nem sempre o que víamos na documentação oficial possuia correspondente em nosso binding. 
 
Por último, um desafio que tivemos foi encontrar um professor de física disponível para nos auxiliar na elaboração do protótipo do sistema. Ficamos muito felizes quando após algumas semanas o bacharel em física e aluno do BCC João Kerr veio a uma de nossas reuniões, a convite do professor Coelho.

\subsection{Disciplinas mais relevantes}

\begin{itemize}

\item MAC0110   Introdução à Computação : Embora já tivesse contato com programação no ensino técnico, foi nesta disciplina que passei a conhecer e utilizar boas práticas de programação. Além disso, estudei algoritmos famosos e interessantes (por exemplo os de ordenação) que estimularam-me para as disciplinas que viriam a seguir.

\item FAP0126 	Física I : É a grande motivação deste trabalho. Os conceitos aprendidos nesta disciplina estão por todo nosso código e nas simulações produzidas. Com o Physimulation, tentamos unir o que vimos nesta disciplina com a computação.
 
\item MAC0122 	Princípios de Desenvolvimento de Algoritmos : O maior contato com algoritmos, dos mais simples e elegantes aos mais complexos, foi fundamental para minha formação. Primeiro porque me desafiou em certos momentos - e consequentemente me deu coragem para analisar ou implementar futuros algoritmos - e em segundo lugar pois deu-me a confiança de que gostaria de seguir carreira em computação.
 
\item MAC0211 	Laboratório de Programação I : Esta disciplina foi interessante por dois motivos: pelo estímulo ao trabalho em equipe e por nos apresentar conceitos e ferramentas relacionadas a qualidade de software, como o Doxygen para documentação de código. Foi nesta disciplina que aprendi o que era um Makefile!

\item MAC0323 	Estruturas de Dados : Essencial para minha formação como cientista da computação. As estruturas aprendidas nesta disciplina - como listas ligadas e árvores - são muito comuns na programação, mesmo no mercado. Possuem vantagens e desvantagens entre si e o conhecimento de suas propriedades assim como os algoritmos adequados para manipulá-las foram muito importantes para mim.

\item MAC0420 	Introdução a Computação Gráfica : Outro forte motivador para nosso trabalho. Nesta disciplina tivemos como exercício-programa a simulação de um jogo de bilhar em três dimensões. Foi uma das experiências mais gratificantes do meu BCC, pois minha dupla e eu aplicamos física em um código simples em C com algumas bibliotecas gráficas e de repente tínhamos uma simulação razoável do que ocorre na vida real. Foi quando percebi que com poucos conceitos de física podíamos reproduzir muitos fenômenos naturais, como colisões e dissipação de energia. Percebi também o quanto estes resultados me motivavam a estudar mais, tanto física quanto computação.

\item FAP0137 	Física II : Os tópicos desta disciplina não foram o foco deste projeto, mas foram grandes motivadores para nosso trabalho. Assim como em Física I, houve pouca contextualização do que foi estudado com o curso de computação. No futuro, temas como relatividade restrita poderão se tornar bem mais simples de se entender por meio de animações criadas pelo próprio usuário, utilizando nosso simulador.
 
\item MAC0332 	Engenharia de Software : na área de computação, um dos conceitos mais recorrentes em qualquer projeto de longo prazo é ciclo de vida de um \textit{software}. Este era o tópico mais discutido na disciplina, tornando-a fundamental para o aluno de computação. Outro aspecto a destacar é a prática de trabalho em equipe.

\item MAC0338 	Análise de Algoritmos: difícil descrever em poucas linhas o quanto esta disciplina é importante para o aluno de computação. Além do desempenho ser uma preocupação constante e necessária a qualquer programador, o aprendizado nesta disciplina é uma das minhas bases sólidas como desenvolvedor. É uma das matérias que quero aprofundar meus conhecimentos durante o mestrado.

\item MAC0446 	Princípios de Interação Homem-computador : uma das disciplinas mais legais para o aluno que está preocupado com a usabilidade de seu sistema. Os conceitos aprendidos estão por todo o trabalho, assim como eu outros projetos que participei ou fui responsável.

\end{itemize}
\subsection{Estudos futuros} 
Sem dúvida os tópicos de estudo mais importantes para a continuação deste trabalho são as disciplinas de Física I e II para o BCC. Quanto maior o conhecimento das leis e forças físicas presentes no mundo real, melhor serão as simulações e consequentemente as animações geradas.

Em segundo lugar, seria interessante uma análise de qual das alternativas a seguir tem uma melhor relação custo-benefício, visando a atualização do projeto com a versão mais nova do Chipmunk: A) migrar nosso projeto de Ruby para C++ e usar diretamente a versão original do Chipunk, sem bindings; ou B) atualizar o binding em Ruby adicionando os métodos e funcionalidades da versão mais recente em C++.

Por último, mas não menos importante, um estudo de paradigmas que proporcionem mais usabilidade ao usuário, substituindo o preenchimento obrigatório de formulários para criação de objetos físicos. Ex: \textit{drag-and-drop} do mouse para "arrastar" as formas geométricas, fornecendo os valores de massa, coeficientes de elasticidade e atrito \textit{a posteriori} (após o objeto já estar na tela).


\section{Rafael Miyagawa}
