
Com o desenvolvimento do Physimulation e as integrações com exercícios-programas passamos por um grande aprendizado nas áreas de simulação e animação, programação - principalmente por utilizarmos Ruby, uma linguagem nova para nós - e física. Nossa intenção é que boa parte deste aprendizado seja repassado para os alunos de física e computação em geral, assim como a motivação para o estudo de disciplinas teóricas do IME. \\

Para os alunos interessados em discretização de simulações, seja para trabalhos acadêmicos ou para jogos que envolvam algum tipo de simulação física, recomendamos a leitura da seção \ref{discretizacao}, em que mostramos um pouco do que aprendemos em relação a tempo de simulação. Dependendo do objetivo do aluno, este tempo deverá ser fixo ou variável. Além disso, há a questão da simulação ser ou não em tempo real: há casos em que é mais importante ter uma animação com resultados precisos e realistas do que uma resposta imediata, mesmo que o processamento demande horas de cálculos. \\

O estudo de alguns conceitos e algoritmos de geometria computacional para detecção e tratamento de colisões também foi bastante interessante. A seção \ref{colisoes} possui referências para o aluno que desejar um conhecimento mais profundo desta área, que é inclusive uma das disciplinas eletivas oferecidas pelo IME atualmente. (TODO confirmar) \\

Como visto na seção \ref{physimulation}, o código do Physimulation foi construído com uma preocupação constante: ser fácil de entender e de modificar. Esperamos que com isso os alunos se sintam mais à vontade para utilizar nosso trabalho ou mesmo alterá-lo, dando continuidade ao projeto. Alguns trabalhos futuros estão descritos na parte subjetiva desta monografia. \\

Assim, esperamos que o Physimulation seja de fato uma contribuição, mesmo que pequena, à questão de contextualização das disciplinas do IME. Como foi dito na apresentação do trabalho, nossa expectativa é que esta iniciativa dê resultados a médio e longo prazo, tendo como horizonte a utilização do Physimulation em salas de aula pelo próprio professor e em conjunto com exercícios teóricos e EP's.
