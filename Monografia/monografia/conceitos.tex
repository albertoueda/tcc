Definições de conceitos, frameworks e linguagem utilizada.

\subsection{Linguagem e Plataformas}

\subsubsection{Ruby}

A linguagem Ruby é uma linguagem dinâmica orientada a objetos. Ruby é responsável pela lógica dos ambientes físicos e pela interação do usuário de interface com
o simulador físico.

\subsubsection{Chipmunk}

Chipmunk é responsável pela mecânica da simulação de física. É feito em C e possui uma extensão para a linguagem Ruby. Na versão 6 é possível selecionar
os algoritmos da Broad Phase. Os algoritmos são Sweep and Prune, AABB Tree e Spatial Hashing. Para detecção de colisão, o chipmunk utiliza Separating Axis Theorem.

\subsubsection{Gosu e Chingu}

Gosu e Chingu são ferramentas, em ruby, para criar interfaces para jogos. Utilizado para mostrar o ambiente físico.

\subsubsection{Glade}

Glade é uma outra ferramenta, do linux, para criação de GUI(Graphical User Interface). A interface inicial é feita com o glade.

\subsection{Configuração do ambiente de desenvolvimento}

Colocar o texto disponível em README.md!

\subsection{Conceitos}

\subsubsection{Tempo de simulação}

Tempo de simulação é o tempo que é utilizado no simulador do ambiente físico, ou seja, o tempo em que os objetos físicos irão se interagir.
COMPLETAR!
\paragraph{Fixo}
\paragraph{Variável}
\paragraph{Semi-Fixo}

\subsubsection{Broad Phase}
É a fase utilizado para filtrar os objetos físicos que deverão passar pelo algoritmo de detecção de colisão. Essa fase é importante para a otimização da simulação.
COMPLETAR!

\paragraph{Algoritmo Sweep and Prune}
\paragraph{AABB Tree}
\paragraph{Spatial Hashing}

\subsubsection{Detecção de colisão - SAT Theorem (Separatinh Axis Theorem}
\subsection{Conceitos físicos do chipmunk}
COMPLETAR!
